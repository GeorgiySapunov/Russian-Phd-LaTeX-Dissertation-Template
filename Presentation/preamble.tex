\begin{frame}[noframenumbering,plain]
    \setcounter{framenumber}{1}
    \maketitle
\end{frame}

\begin{frame}
    \frametitle{Положения, выносимые на защиту}
    \begin{itemize}
%        \item Результаты расчёта этого путём таким-то.
%        \item Результаты разработки того.
%        \item И ещё \dots
%        \item \dots пару пунктов.
        \item	Установлено, что метод капельной молекулярно-пучковой эпитаксии позволяет синтезировать на поверхности Si эпитаксиальные GaN наноостровки с метастабильной структурой ZB, \{111\} грани которых могут служить центрами зародышеобразования для последующего роста нанообъектов в форме триподов;
        \item Выявлено влияние затравок (Si\textsubscript{3}N\textsubscript{4}, AlN, GaO\textsubscript{x} и субмонослойные слои Ga) на морфологию и оптические свойства ННК GaN;
        \item Установлены условия синтеза эпитаксиальных наночастиц GaAs диаметром от 200~\si{\nano\metre} до 2~\si{\micro\metre};
        \item Установлено, что зарождение паразитных островков при росте ННК GaP с низким аспектным отношением длина/толщина может быть подавлено понижением потока Ga на начальной стадии роста.
    \end{itemize}
\end{frame}
\note{
    Проговариваются вслух положения, выносимые на защиту
}

\begin{frame}
    \frametitle{Содержание}
    \tableofcontents
\end{frame}
\note{
    Работа состоит из четырёх глав.

    \medskip
    В первой главе \dots

    Во второй главе \dots

    Третья глава посвящена \dots

    В четвёртой главе \dots
}
