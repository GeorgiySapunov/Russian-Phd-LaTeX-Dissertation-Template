
{\actuality} Развитие технологий производства полупроводниковых приборов на
основе кремния (Si) привело к  доминации этого материала в интегральной
электронике и фотовольтаике. При этом он малопригоден для создания
светоизлучающих приборов из-за непрямой зонной структуры и низкой вероятности
излучательных переходов. Развитие технологий монолитной интеграции
оптоэлектронных элементов на Si может снизить стоимость производства
светоизлучающих приборов (за счёт отказа от дорогостоящих
A\textsuperscript{III}B\textsuperscript{V} подложек) и заменить электронные
компоненты интегральных схем на оптические (для снижения энергопотребления и
изоляции их компонентов друг от друга).

Монолитная интеграция тонкоплёночных структур
A\textsuperscript{III}B\textsuperscript{V} на Si связана с проблемами
несоответствия параметров кристаллических решёток и различием их симметрии:
зарождение A\textsuperscript{III}B\textsuperscript{V} на Si возможно с
различной полярностью, что может приводить к образованию антифазных областей.
Их границы~--- эффективные центры безызлучательной рекомбинации
\cite{Takagi1998}. Образование антифазных областей возможно при росте
материалов с симметрией решётки ниже, чем у подложки. Например, при
формировании соединения с двумя различными атомами в примитивной решётке на
поверхности одноэлементного материала (арсенид галлия (GaAs) на подложке
германия (Ge) или фосфид галлия (GaP) на подложке Si).

Продолжающееся исследования эпитаксиального роста
A\textsuperscript{III}B\textsuperscript{V} слоев на Si (в большей степени это
касается твёрдых растворов на основе GaP) стимулирует развитие технологий
синтеза функциональных слоев высокого кристаллического качества. Однако, оно
ещё не позволяет достичь достаточно высокого квантового выхода излучения для их
приборного применения.

Альтернативный подход формирования оптоэлектронных элементов на Si включает
синтез A\textsuperscript{III}B\textsuperscript{V} эпитаксиальных наноструктур.
В отличие от тонкоплёночных гетероструктур, они имеют малую площадь интерфейса
с подложкой и высокое отношение поверхности к объёму \cite{Bolshakov2013,
Tchernycheva2007}. Это обеспечивает эффективную релаксацию напряжений и низкую
концентрацию структурных дефектов даже в системах с высоким рассогласованием
кристаллической решётки \cite{Samsonenko2011}.

Более того, наноструктуры предоставляют дополнительные возможности для зонного
проектирования~--- в них удаётся стабилизировать кристаллическую структуру,
невозможную в объёмном материале при нормальных условиях \cite{Mohseni2009},
что расширяет функциональные возможности приборных гетероструктур
\cite{Spirkoska2009}. Например, GaP с кубической структурой сфалерит (ZB)~---
непрямозонный полупроводник, а с гексагональной структурой вюрцит (WZ)~---
прямозонный полупроводник с шириной запрещённой зоны
2,18--2,25~\si{\electronvolt}, а значит может найти применение в производстве
жёлто-зелёных светодиодов \cite{Assali2013}. Вместе с тем, из-за симметрии
структуры WZ симметричны при двойниковании по плоскостям \{111\}, в отличии от
структуры ZB, что обеспечивает более высокое кристаллическое совершенство
синтезируемых наноструктур.

Возможность управлять формой наночастиц и их размером в диапазоне от
микрометров до единиц нанометров позволяет формировать структуры с оптическим
или электронным ограничением, что находит применение в оптоэлектронных
приборах. Например, для миниатюризации генераторов лазерного излучения
требуется уменьшение оптически активной области резонатора. Во множестве работ
демонстрируется, что в ННК может формироваться резонансная стоячая оптическая
волна вдоль оси ННК с отражением от верхней и нижней граней. На основе ННК
резонаторов продемонстрирована генерация лазерного излучения с оптической и
электрической накачкой \cite{Eaton2016}.

В общем случае форма, размер и поверхностная плотность наноструктур влияет на
их взаимодействие со светом и определяет эффективность вывода или захвата
излучения. Этот эффект находит применение в просветляющих покрытиях солнечных
элементов \cite{Mozharov2015b, Krogman2005} и определяет диаграмму
направленности светоизлучающих приборов на основе массивов наногетероструктур
\cite{Eaton2016}.

Несмотря на достигнутые результаты, остаются не до конца изучены закономерности
формирования самоорганизованных наноструктур, устанавливающие взаимосвязь
морфологии, кристаллической структуры и оптических свойств с ростовыми
условиями и предварительной подготовкой подложки. Изучение данных
закономерностей позволит развить методы синтеза самоорганизованных наноструктур
для применения в приборах оптоэлектроники.

%\ifsynopsis
%Этот абзац появляется только в~автореферате.
%Для формирования блоков, которые будут обрабатываться только в~автореферате,
%заведена проверка условия \verb!\!\verb!ifsynopsis!.
%Значение условия задаётся в~основном файле документа (\verb!synopsis.tex! для
%автореферата).
%\else
%Этот абзац появляется только в~диссертации.
%Через проверку условия \verb!\!\verb!ifsynopsis!, задаваемого в~основном файле
%документа (\verb!dissertation.tex! для диссертации), можно сделать новую
%команду, обеспечивающую появление цитаты в~диссертации, но~не~в~автореферате.
%\fi

% {\progress}
% Этот раздел должен быть отдельным структурным элементом по
% ГОСТ, но он, как правило, включается в описание актуальности
% темы. Нужен он отдельным структурынм элемементом или нет ---
% смотрите другие диссертации вашего совета, скорее всего не нужен.

{\aim} данной работы является исследование закономерностей формирования и
развитие методов синтеза самоорганизованных массивов эпитаксиальных
A\textsuperscript{III}B\textsuperscript{V} наноструктур на Si для применения в
приборах оптоэлектроники.

Для~достижения поставленной цели необходимо было решить следующие {\tasks}:
\begin{enumerate}[beginpenalty=10000] % https://tex.stackexchange.com/a/476052/104425
  \item	Исследовать закономерности формирования, морфологию и кристаллическую структуру триподов GaN на подложках Si(111);
  \item	Исследовать влияние подготовки поверхности подложки Si(111) и состава буферного слоя на морфологию и фотолюминесценцию нитевидных нанокристаллов GaN;
  \item Исследовать закономерности формирования, морфологию и фотолюминесценцию GaAs наночастиц на подложках Si(111);
  \item Исследовать закономерности формирования, морфологию и кристаллическую структуру нитевидных нанокристаллов GaP.
\end{enumerate}


{\novelty}
\begin{enumerate}[beginpenalty=10000] % https://tex.stackexchange.com/a/476052/104425
  \item Впервые показано, что рост на наноостровках GaN на Si(111) может вести к формированию GaN наноструктур в форме трипода;
  \item Впервые исследовано влияние ростовых условий на морфологию массива триподов GaN, полученных методом капельной эпитаксии;
  \item Впервые показано, как подготовка ростовой поверхности подложки Si(111) (нитридация в активированном азоте с образованием слоя SiN\textsubscript{x}, нанесение буферного слоя GaO\textsubscript{x}, формирование затравочных островков AlN и GaN, нанесение смачивающих слоев Ga эквивалентной толщиной 0,3 и 0,6~монослоя) влияет на морфологию и оптические свойства ННК GaN;
  \item	Впервые исследованы основные закономерности формирования, оптические свойства и морфология наночастиц GaAs на подложках Si(111);
  \item	Предложен метод синтеза эпитаксиальных наночастиц GaAs диаметром от 200~\si{\nano\meter} до 2~\si{\micro\metre};
  \item	Впервые исследовано влияние условий роста на морфологию массива ННК GaP;
  \item Предложен двухстадийный метод формирования массива ННК GaP высокой поверхностной плотности и толщины вертикальных ННК.
\end{enumerate}

{\influence} Результаты работы могут быть использованы для создания эпитаксиальных функциональных гетероструктур на Si(111) для приборов оптоэлектроники.

{\methods}
\begin{enumerate}[beginpenalty=10000]
  \item Метод молекулярно-пучковой эпитаксии применён для синтеза наноструктур;
  \item Метод дифракции быстрых электронов применён для исследования кристаллической структуры и морфологии наноструктур;
  \item Метод растровой и просвечивающей электронной микроскопии применён для исследования кристаллической структуры и морфологии наноструктур;
  \item Метод атомно-силовой микроскопии применён для исследования морфологии наноструктур;
  \item Метод спектроскопии фотолюминесценции применён для исследования оптических свойств наноструктур;
  \item Метод спектроскопии комбинационного рассеяния света применён для исследования кристаллической структуры наноструктур.
\end{enumerate}

{\defpositions}
\begin{enumerate}[beginpenalty=10000] % https://tex.stackexchange.com/a/476052/104425
  \item	Установлено, что метод капельной молекулярно-пучковой эпитаксии позволяет синтезировать на поверхности Si эпитаксиальные GaN наноостровки с метастабильной структурой ZB, \{111\} грани которых могут служить центрами зародышеобразования для последующего роста нанообъектов в форме триподов;
  \item Выявлено влияние затравок (Si\textsubscript{3}N\textsubscript{4}, AlN, GaO\textsubscript{x} и субмонослойные слои Ga) на морфологию и оптические свойства ННК GaN;
  \item Установлены условия синтеза эпитаксиальных наночастиц GaAs диаметром от 200~\si{\nano\metre} до 2~\si{\micro\metre};
  \item Установлено, что зарождение паразитных островков при росте ННК GaP с низким аспектным отношением длина/толщина может быть подавлено понижением потока Ga на начальной стадии роста.
\end{enumerate}

{\reliability} полученных результатов обеспечивается подтверждается воспроизводимостью экспериментальных данных и соответствием результатов, полученных с применением взаимодополняющих методов исследования.


{\probation}
Основные результаты работы докладывались~на следующих конференциях:
\begin{enumerate}[beginpenalty=10000]
  \item Международная школа-конференция “Saint-Petersburg OPEN 2016”;
  \item XVIII Всероссийская молодежная конференция по физике полупроводников и наноструктур, полупроводниковой опто- и наноэлектронике;
  \item Международная конференция ФизикА.СПб/2017;
  \item Международная школа-конференция “Saint-Petersburg OPEN 2018”;
  \item Международная конференция ФизикА.СПб/2019.
\end{enumerate}

{\contribution} Автор принимал активное участие в планировании экспериментов, эпитаксиальном синтезе и обработке экспериментальных результатов.

\ifnumequal{\value{bibliosel}}{0}
{%%% Встроенная реализация с загрузкой файла через движок bibtex8. (При желании, внутри можно использовать обычные ссылки, наподобие `\cite{vakbib1,vakbib2}`).
    {\publications} Основные результаты по теме диссертации изложены
    в~27~печатных изданиях,
   27 из которых изданы в журналах, рекомендованных ВАК.
}%
{%%% Реализация пакетом biblatex через движок biber
    \begin{refsection}[bl-author, bl-registered]
        % Это refsection=1.
        % Процитированные здесь работы:
        %  * подсчитываются, для автоматического составления фразы "Основные результаты ..."
        %  * попадают в авторскую библиографию, при usefootcite==0 и стиле `\insertbiblioauthor` или `\insertbiblioauthorgrouped`
        %  * нумеруются там в зависимости от порядка команд `\printbibliography` в этом разделе.
        %  * при использовании `\insertbiblioauthorgrouped`, порядок команд `\printbibliography` в нём должен быть тем же (см. biblio/biblatex.tex)
        %
        % Невидимый библиографический список для подсчёта количества публикаций:
        \printbibliography[heading=nobibheading, section=1, env=countauthorvak,          keyword=biblioauthorvak]%
        \printbibliography[heading=nobibheading, section=1, env=countauthorwos,          keyword=biblioauthorwos]%
        \printbibliography[heading=nobibheading, section=1, env=countauthorscopus,       keyword=biblioauthorscopus]%
        \printbibliography[heading=nobibheading, section=1, env=countauthorconf,         keyword=biblioauthorconf]%
        \printbibliography[heading=nobibheading, section=1, env=countauthorother,        keyword=biblioauthorother]%
        \printbibliography[heading=nobibheading, section=1, env=countregistered,         keyword=biblioregistered]%
        \printbibliography[heading=nobibheading, section=1, env=countauthorpatent,       keyword=biblioauthorpatent]%
        \printbibliography[heading=nobibheading, section=1, env=countauthorprogram,      keyword=biblioauthorprogram]%
        \printbibliography[heading=nobibheading, section=1, env=countauthor,             keyword=biblioauthor]%
        \printbibliography[heading=nobibheading, section=1, env=countauthorvakscopuswos, filter=vakscopuswos]%
        \printbibliography[heading=nobibheading, section=1, env=countauthorscopuswos,    filter=scopuswos]%
        %
        \nocite{*}%
        %
        {\publications} Основные результаты по теме диссертации изложены в~\arabic{citeauthor}~печатных изданиях,
        \arabic{citeauthorvak} из которых изданы в журналах, рекомендованных ВАК\sloppy%
        \ifnum \value{citeauthorscopuswos}>0%
            , \arabic{citeauthorscopuswos} "--- в~периодических научных журналах, индексируемых Web of~Science и Scopus\sloppy%
        \fi%
        \ifnum \value{citeauthorconf}>0%
            , \arabic{citeauthorconf} "--- в~тезисах докладов.
        \else%
            .
        \fi%
        \ifnum \value{citeregistered}=1%
            \ifnum \value{citeauthorpatent}=1%
                Зарегистрирован \arabic{citeauthorpatent} патент.
            \fi%
            \ifnum \value{citeauthorprogram}=1%
                Зарегистрирована \arabic{citeauthorprogram} программа для ЭВМ.
            \fi%
        \fi%
        \ifnum \value{citeregistered}>1%
            Зарегистрированы\ %
            \ifnum \value{citeauthorpatent}>0%
            \formbytotal{citeauthorpatent}{патент}{}{а}{}\sloppy%
            \ifnum \value{citeauthorprogram}=0 . \else \ и~\fi%
            \fi%
            \ifnum \value{citeauthorprogram}>0%
            \formbytotal{citeauthorprogram}{программ}{а}{ы}{} для ЭВМ.
            \fi%
        \fi%
        % К публикациям, в которых излагаются основные научные результаты диссертации на соискание учёной
        % степени, в рецензируемых изданиях приравниваются патенты на изобретения, патенты (свидетельства) на
        % полезную модель, патенты на промышленный образец, патенты на селекционные достижения, свидетельства
        % на программу для электронных вычислительных машин, базу данных, топологию интегральных микросхем,
        % зарегистрированные в установленном порядке.(в ред. Постановления Правительства РФ от 21.04.2016 N 335)
    \end{refsection}%
    \begin{refsection}[bl-author, bl-registered]
        % Это refsection=2.
        % Процитированные здесь работы:
        %  * попадают в авторскую библиографию, при usefootcite==0 и стиле `\insertbiblioauthorimportant`.
        %  * ни на что не влияют в противном случае

    \end{refsection}%
        %
        % Всё, что вне этих двух refsection, это refsection=0,
        %  * для диссертации - это нормальные ссылки, попадающие в обычную библиографию
        %  * для автореферата:
        %     * при usefootcite==0, ссылка корректно сработает только для источника из `external.bib`. Для своих работ --- напечатает "[0]" (и даже Warning не вылезет).
        %     * при usefootcite==1, ссылка сработает нормально. В авторской библиографии будут только процитированные в refsection=0 работы.
}

%При использовании пакета \verb!biblatex! будут подсчитаны все работы, добавленные
%в файл \verb!biblio/author.bib!. Для правильного подсчёта работ в~различных
%системах цитирования требуется использовать поля:
%\begin{itemize}
%        \item \texttt{authorvak} если публикация индексирована ВАК,
%        \item \texttt{authorscopus} если публикация индексирована Scopus,
%        \item \texttt{authorwos} если публикация индексирована Web of Science,
%        \item \texttt{authorconf} для докладов конференций,
%        \item \texttt{authorpatent} для патентов,
%        \item \texttt{authorprogram} для зарегистрированных программ для ЭВМ,
%        \item \texttt{authorother} для других публикаций.
%\end{itemize}
%Для подсчёта используются счётчики:
%\begin{itemize}
%        \item \texttt{citeauthorvak} для работ, индексируемых ВАК,
%        \item \texttt{citeauthorscopus} для работ, индексируемых Scopus,
%        \item \texttt{citeauthorwos} для работ, индексируемых Web of Science,
%        \item \texttt{citeauthorvakscopuswos} для работ, индексируемых одной из трёх баз,
%        \item \texttt{citeauthorscopuswos} для работ, индексируемых Scopus или Web of~Science,
%        \item \texttt{citeauthorconf} для докладов на конференциях,
%        \item \texttt{citeauthorother} для остальных работ,
%        \item \texttt{citeauthorpatent} для патентов,
%        \item \texttt{citeauthorprogram} для зарегистрированных программ для ЭВМ,
%        \item \texttt{citeauthor} для суммарного количества работ.
%\end{itemize}
% Счётчик \texttt{citeexternal} используется для подсчёта процитированных публикаций;
% \texttt{citeregistered} "--- для подсчёта суммарного количества патентов и программ для ЭВМ.
