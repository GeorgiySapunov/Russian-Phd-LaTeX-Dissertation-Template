\begin{frame}[noframenumbering,plain]
    \setcounter{framenumber}{1}
    \maketitle
\end{frame}

\begin{frame}
    \frametitle{Положения, выносимые на защиту}
    \begin{itemize}
%        \item Результаты расчёта этого путём таким-то.
%        \item Результаты разработки того.
%        \item И ещё \dots
%        \item \dots пару пунктов.
\item Установлено, что в процессе капельной молекулярно-пучковой эпитаксии на
поверхности Si(111) могут формироваться GaN наноостровки с метастабильной
структурой сфалерита. При последующем росте GaN
% в условиях пониженной температуры (\(\approx 690\)~\si{\degreeCelsius}) и
% пониженного потока Ga (\(\approx 1 \cdot 10^{-8}\)~\si{\torr})
на \{111\} гранях наноостровков могут зарождаться наклонные наноколонны со
структурой вюрцита, что приводит к формированию наноструктур в форме
трипода.
\item Показано, что модификация поверхности поверхности Si(111) путем формирования буферных слоёв SiN\textsubscript{x}, AlN, GaO\textsubscript{x}, Ga-индуцированной поверхностной реконструкции при молекулярно-пучковой эпитаксии GaN позволяет контролировать морфологию массива ННК. Наиболее однородный по длине и диаметру ННК массив с минимальной поверхностной плотностью
формируется на затравке AlN. Массив с наибольшей поверхностной плотностью~--- на затравке, полученной в результате нитридации реконструкции
Si\((111)\sqrt{3}\)\(\times\)\(\sqrt{3} - R30\si{\degree} - \text{Ga}\).
\item Выявлено, что на поверхности Si\((111)\) с
удалённым поверхностным окислом, могут эпитаксиально формироваться наночастицы GaAs по механизму пар\,--\,жидкость\,--\,кристалл. Рост при пониженном
отношения молекулярных потоков As/Ga или повышенной температуре приводит к
увеличению диаметра наночастиц в диапазоне от сотен нанометров до единиц
микрометров.
\item Установлено, что диаметр самокаталитических ННК GaP определяется
отношением молекулярных потоков и температурой. Изменением данных
параметров в процессе синтеза не нарушает механизм роста
пар\,--\,жидкость\,--\,кристалл, что обеспечивает независимое управление
аспектным отношением и поверхностной плотностью ННК GaP.
    \end{itemize}
\end{frame}
\note{
    Проговариваются вслух положения, выносимые на защиту
}

\begin{frame}
    \frametitle{Содержание}
    \tableofcontents
\end{frame}
\note{
    Работа состоит из четырёх глав.

    \medskip
    В первой главе \dots

    Во второй главе \dots

    Третья глава посвящена \dots

    В четвёртой главе \dots
}
