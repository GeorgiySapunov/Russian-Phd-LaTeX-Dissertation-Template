\begin{frame}
    \frametitle{Научная новизна}
    \begin{itemize}
%        \item Впервые реализован \dots
%        \item Разработана программа \dots
%        \item Впервые проведён анализ \dots
%        \item Предложена схема \dots
\item Впервые показано, что рост на наноостровках GaN на Si(111) может вести к формированию GaN наноструктур в форме трипода;
\item Впервые исследовано влияние ростовых условий на морфологию массива триподов GaN, полученных методом капельной эпитаксии;
\item Впервые показано, как подготовка ростовой поверхности подложки Si(111) (нитридация в активированном азоте с образованием слоя SiN\textsubscript{x}, нанесение буферного слоя GaO\textsubscript{x}, формирование затравочных островков AlN и GaN, нанесение смачивающих слоев Ga эквивалентной толщиной 0,3 и 0,6~монослоя) влияет на морфологию и оптические свойства ННК GaN;
\item	Впервые исследованы основные закономерности формирования, оптические свойства и морфология наночастиц GaAs на подложках Si(111);
\item	Предложен метод синтеза эпитаксиальных наночастиц GaAs диаметром от 200~\si{\nano\meter} до 2~\si{\micro\metre};
\item	Впервые исследовано влияние условий роста на морфологию массива ННК GaP;
\item Предложен двухстадийный метод формирования массива ННК GaP высокой поверхностной плотности и толщины вертикальных ННК.
    \end{itemize}
\end{frame}
\note{
    Проговаривается вслух научная новизна
}

\begin{frame}
    \frametitle{Научная и практическая значимость}
Результаты работы могут быть использованы для разработки функциональных эпитаксиальных гетероструктур на Si для приборов фотоники, в там числе:
    \begin{itemize}
	\item светодиодов и лазерных диодов;
	\item оптических резонаторов и волноводов;
	\item просветляющих покрытий фотоэлементов.
    \end{itemize}
\end{frame}
\note{
    Проговариваются вслух научная и практическая значимость
}
%
%\begin{frame}
%    \frametitle{Свидетельство о регистрации программы}
%    \begin{figure}[h]
%        \centering
%        \includegraphics[height=0.7\textheight]{registration}
%    \end{figure}
%\end{frame}
%\note{
%    Получено свидетельство о регистрации разработанной программы \textsc{Hello~world™}.
%}
%
%\begin{frame}
%    \frametitle{Акт о внедрении}
%    \begin{figure}[h]
%        \centering
%        \fbox{
%            \begin{minipage}[t]{0.4\linewidth}
%                \includegraphics[width=\linewidth]{implementation}
%            \end{minipage}
%        }
%    \end{figure}
%\end{frame}
%\note{
%    Получен акт о внедрении.
%}

\begin{frame}
	\frametitle{Участие в конференциях}
	\begin{itemize}
		\item Международная конференция ФизикА.СПб/2020;
		\item Международная конференция ФизикА.СПб/2019;
		\item Международная школа-конференция “Saint-Petersburg OPEN 2018”;
		\item Международная конференция ФизикА.СПб/2017;
		\item XVIII Всероссийская молодежная конференция по физике полупроводников и наноструктур, полупроводниковой опто- и наноэлектронике;
		\item Международная школа-конференция “Saint-Petersburg OPEN 2016”.
	\end{itemize}
\end{frame}
\note{
	Работа была представлена на ряде конференций.
}

\begin{frame} % публикации на одной странице
% \begin{frame}[t,allowframebreaks] % публикации на нескольких страницах
    \frametitle{Основные публикации}
    \nocite{koval2020b}%
    \nocite{Koval2020}%
    \nocite{Shugurov2020}%
    \nocite{Bolshakov2019}%
    \nocite{Bolshakov2019a}%
    \nocite{Sapunov2019b}%
    \nocite{Fedorov2018a}%
    \nocite{Bolshakov2018}%
        \ifnumequal{\value{bibliosel}}{0}{
        \insertbiblioauthor
    }{
        \printbibliography%
    }
\end{frame}
\note{
    Результаты работы опубликованы в N печатных изданиях,
    в~т.\:ч. M реферируемых изданиях.
}

%\begin{frame}[plain, noframenumbering] % последний слайд без оформления
%    \begin{center}
%        \Huge
%        Спасибо за внимание!
%    \end{center}
%\end{frame}
