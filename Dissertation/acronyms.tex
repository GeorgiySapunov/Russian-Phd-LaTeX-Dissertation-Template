\chapter*{Список сокращений и условных обозначений} % Заголовок
\addcontentsline{toc}{chapter}{Список сокращений и условных обозначений}  % Добавляем его в оглавление
% при наличии уравнений в левой колонке значение параметра leftmargin приходится подбирать вручную
\begin{description}[align=right,leftmargin=3.5cm]
   \item[ННК] нитевидный нанокристалл;
   \item[WZ] wurtzite, гексагональная кристаллическая структура типа вюрцита;
   \item[ZB] zincblende, кубическая кристаллическая структура типа цинковой обманки;
   \item[ПЖК] механизм роста пар\,--\,жидкость\,--\,кристалл;
   \item[МПЭ] молекулярно-пучковая эпитаксия;
   \item[ПА-МПЭ] молекулярно-пучковая эпитаксия с плазменной активацией азота;
   \item[ВЧ] высокочастотный;
   \item[ЭДП] эквивалентное давление пучка;
   \item[ДБЭ] дифракция быстрых электронов;
   \item[РЭМ] растровый электронный микроскоп;
   \item[ПЭМ] просвечивающий электронный микроскоп;
   \item[ПРЭМ] просвечивающий растровый электронный микроскоп;
   \item[ВРЭМ] высокоразрешающая электронная микроскопия;
   \item[АСМ] атомно-силовая микроскопия;
   \item[ФЛ] фотолюминесценция;
   \item[КРС] комбинационное рассеяния света;
   \item[БПФ] быстрое преобразование Фурье;
   \item[ПХГФО] плазмохимическое осаждение из газовой фазы;
   \item[TO] transverse optical, поперечный оптический;
   \item[LO] longitudinal optical, продольный оптический.

\end{description}
