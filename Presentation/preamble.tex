\begin{frame}[noframenumbering,plain]
    \setcounter{framenumber}{1}
    \maketitle
\end{frame}

\begin{frame}
    \frametitle{Положения, выносимые на защиту}
    \begin{itemize}
%        \item Результаты расчёта этого путём таким-то.
%        \item Результаты разработки того.
%        \item И ещё \dots
%        \item \dots пару пунктов.
  \item	Установлено, что в процессе капельной молекулярно-пучковой эпитаксии на
    поверхности Si(111) могут формироваться GaN наноостровки с метастабильной
    структурой сфалерита. При последующем росте GaN
 % в условиях пониженной температуры (\(\approx 690\)~\si{\degreeCelsius}) и
    % пониженного потока Ga (\(\approx 1 \cdot 10^{-8}\)~\si{\torr})
   на \{111\} гранях наноостровков могут зарождаться наклонные наноколонны со
   структурой вюрцита, что приводит к формированию наноструктур в форме
   трипода.
 \item Показано, что выбор затравок при МПЭ GaN ННК на Si(111) позволяет
   контролировать морфологию массива. Среди затравок SiN\textsubscript{x}, AlN,
   GaO\textsubscript{x}, субмонослойных смачивающих слоёв Ga и затравочных
   островков GaN наиболее однородный по длине и диаметру ННК массив с
   минимальной поверхностной плотностью формируется на затравке AlN. Наименее
   однородный по длине~--- на необработанной подложке с реконструкцией
   Si\((111)7\)\(\times\)\(7\). Массив с наибольшей поверхностной
   плотностью~--- на затравке, полученной в результате нитридации реконструкции
   Si\((111)\sqrt{3}\)\(\times\)\(\sqrt{3} - R30\si{\degree} - \text{Ga}\).
 \item Выявлено, что эпитаксиальный синтез GaAs на поверхности Si\((111)\), с
   которой жидкостным методом удалён поверхностный окисел, может приводить к
   формированию наночастиц GaAs по ПЖК механизму. Рост при пониженном отношения
   молекулярных потоков As/Ga или повышенной температуре приводит к увеличению
   диаметра наночастиц в диапазоне от 200~\si{\nano\metre} до
   2~\si{\micro\metre}.
 \item Установлено, что стабильный диаметр самокаталитических ННК GaP
   определяется отношением молекулярных потоков и температурой. Изменением
   данных параметров в процессе синтеза не нарушает ПЖК механизм роста, что
   дает возможность независимо управлять аспектным отношением и поверхностной
   плотностью ННК GaP.
    \end{itemize}
\end{frame}
\note{
    Проговариваются вслух положения, выносимые на защиту
}

\begin{frame}
    \frametitle{Содержание}
    \tableofcontents
\end{frame}
\note{
    Работа состоит из четырёх глав.

    \medskip
    В первой главе \dots

    Во второй главе \dots

    Третья глава посвящена \dots

    В четвёртой главе \dots
}
