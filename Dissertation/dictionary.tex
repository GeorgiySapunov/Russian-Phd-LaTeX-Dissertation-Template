\chapter*{Словарь терминов}             % Заголовок
\addcontentsline{toc}{chapter}{Словарь терминов}  % Добавляем его в оглавление
\textbf{адатом} : адсорбированный атом;

\textbf{гетероструктура} : искусственная система, имеющая, по крайней мере, один гетеропереход;

\textbf{гетеропереход} : граница раздела между двумя веществами с различными электронными свойствами;

\textbf{эпитаксиальные гетероструктуры} : гетероструктуры со взаимно ориентироваными кристаллические решетки её однородных частей (фаз);

\textbf{гетероэпитаксия} : синтез  эпитаксиальных гетероструктур;

\textbf{твёрдый раствор} : систем соединений переменного состава, в которых атомы различных элементов расположены в общей кристаллической решётке;

\textbf{молекулярно-пучковая эпитаксия} : эпитаксиальный рост из осаждаемых молекул в условиях сверхвысокого вакуума;

\textbf{капельная эпитаксия} : метод формирования эпитаксиальных наночастиц путем кристаллизации металлических капель элемента III группы в потоке элемента V группы;

\textbf{криопанель} : ёмкостью с протекающим в ней жидким азотом, служащая для температурного экранирования горячих источников и уменьшения остаточного давления в вакуумной камере за счет адсорбции;

\textbf{эффузия} : процесс, при котором отдельные молекулы проникают через отверстие без столкновений между собой;

\textbf{эффузионный источник} : источник молекул, длина свободного пробега которых настолько велика, что их взаимодействием в потоке можно пренебречь;

\textbf{поверхностная реконструкция} : процесс перестройки поверхностного слоя кристалла, в результате которой его атомная структура существенным образом отличается от структуры соответствующих атомных плоскостей в объёме кристалла;

\textbf{сверхструктура} : нарушение структуры кристалла, повторяющееся с определенной регулярностью и создающее таким образом новую структуру с другим периодом чередования;

\textbf{светлопольная микроскопия} : вид микроскопии с детектированием всех отраженных или прошедших через образец лучей;

\textbf{темнопольная микроскопия} : вид микроскопии с детектированием только дифрагированных лучей путем введения в фокальную плоскость апертуры;

\textbf{экситон} : связанное водородоподобное состояние, устойчивое благодаря кулоновскому взаимодействию;

\textbf{метод Шираки} : низкотемпературный метод химической подготовки Si подложек к эпитаксиальному росту, предложенный Ясухиро Шираки и Акитоши Ишизака;

\textbf{эквивалентная толщина} : объем осажденного материала, деленный на площадь ростовой поверхности.